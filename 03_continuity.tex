\graphicspath{ {./figuras/03_continuity/} }
\chapter{Funções contínuas}
\label{chp:continuity}

\section{Como Estabelecer um Problema}
\label{sec:howtoproblem}

Quando estamos trabalhando com aplicações do cálculo, os problemas são
frequentemente apresentados de maneira verbal, e é nossa tarefa escrevê-los
em termos matemáticos. Um problema pode ser geralmente descrito por uma
lista de equações e inequações. As próximas duas seções contém vários
exemplos que ilustram o processo de tradução de problemas para a lignuagem
matemática. Os exemplos desta seção usam conceitos de álgebra e geometria,
enquanto que os da próxima seção usam o cálculo.

Às vezes é difícil identificar como começar a solução de um problema que
é contado por meio de uma história. É útil dividir este processo em três
etapas:
\begin{stepanalysis}
\item Desenhe um diagrama, se possível, e rotule todas as quantidades
      envolvidas.
\item Escreva a informação dada como um conjunto de equações e/ou
      inequações.
\item Resolva o problema matemático e interprete a solução encontrada
      para responder o problema original da história.
\end{stepanalysis}

\begin{example}\label{ex:chp3p1ex1}
De acordo com um mapa de tesouro, existe um baú enterrado a oeste de
determinada caverna e a 200 passos de uma árvore. A árvore está a 30
passos a oeste e 40 passos ao norte da caverna. Qual é a distância, em
linha reta, entre o tesouro e a árvore?

A solução para este problema usa a fórmula quadrática, a qual será muito
utilizada ao longo deste curso de cálculo. Nós a revisaremos aqui.

\newdef[fórmula!quadrática]{Fórmula quadrática}: Se $a \ne 0$, então
$$
  ax^2 + bx + c = 0
$$
se, e somente se,
$$
  x = \frac{-b \pm \sqrt{b^2 - 4ac}}{2a}.
$$

Resolveremos o Exemplo~\ref{ex:chp3p1ex1} em três etapas.
\begin{stepanalysis}
\item Desenhe um diagrama e rotule todas as quantidades envolvidas. Na
      Figura~\ref{fig:treasuremap}, colocamos a caverna na origem e
      chamamos de $x$ a distância entre a caverna e o objetivo
      ao longo do eixo dos $x$.
      A árvore está no ponto $(30, 40)$ e o tesouro está no ponto $(x, 0)$.
\includefig{treasuremap}
\item Escreva as informações conhecidas como um sistema de fórmulas. Pela
      Fórmula da Distância, temos
      $$
        200 = \sqrt{(x-30)^2 + (0-40)^2}, \hspace{1.5em} x \ge 0
      $$
      A desigualdade $x \ge 0$ aparece pois o tesouro está a leste da
      caverna.
\item Resolva para $x$. Elevando ambos os lados da Fórmula da Distância
      ao quadrado, temos
      \begin{eqnarray*}
        \fnum{40000} & = & (x - 30)^2 + (0 - 40)^2 = x^2 - 60x + 900 + 1600\\
        & = & x^2 - 60x + 250 \\
        &   & x^2 - 60x + \fnum{37500} = 0
      \end{eqnarray*}
      Para encontrarmos $x$, usamos a Fórmula Quadrática.
      \begin{eqnarray*}
        x & = & \frac{60 \pm \sqrt{(60)^2 - 4(-\fnum{37500}}}{2}
          = \frac{60 \pm \sqrt{\fnum{153600}}}2{} \\
        & = & 30 \pm \sqrt{\fnum{38400}}.
      \end{eqnarray*}
\end{stepanalysis}
\begin{interpretsolution}
  Como $x \ge 0$, rejeitamos a solução negativa. Assim,
  $x = 30 + \sqrt{\fnum{38400}} \sim 226$ passos. O tesouro está a
  aproximadamente 226 passos da caverna.
\end{interpretsolution}
\end{example}

A maioria dos problemas de cálculo envolve duas ou mais variáveis.

\begin{example}
Um homem de $1,8\si{m}$ de altura está próximo de um poste cuja lâmpada
está a $3\si{m}$ de altura. Encontre o comprimento de sua sombra em função
da sua distância ao poste.
\begin{stepanalysis}
\item Desenhe um diagrama e rotule todas as quantidades envolvidas. Na
      Figura~\ref{fig:exlamppost}, sejam
      \begin{eqnarray*}
        x & = & \text{distância do homem ao poste,} \\
        s & = & \text{comprimento de sua sombra}.
      \end{eqnarray*}
\includefig{exlamppost}
\item Escreva as informações conhecidas como um sistema de fórmulas. Por
      semelhança de triângulos, temos
      $$
        \frac{s}{1,8} = \frac{s + x}{3}, \hspace{1.5em} x \ge 0.
      $$
      A desigualdade $x \ge 0$ aparece pois a distância não pode ser
      negativa.
\item Resolva para $s$ em função de $x$.
      \begin{eqnarray*}
        3s & = & 1,8s + 1,8x, \\
        1,2s & = & 1,8x, \\
         12s & = & 18x, \\
        s & = & \frac{18}{12}x = \frac{3}{2}x.
      \end{eqnarray*}
\end{stepanalysis}
\begin{interpretsolution} $s = \frac{3}{2}x$, \hspace{1.5ex} $x \ge 0$.

  O domínio da função é $[0, \infty)$. O comprimento da sombra é
  $\frac{3}{2}$ vezes a distância do homem ao poste. Neste problema, $x$
  é a variável independente e $s$ depende de $x$.
\end{interpretsolution}
\end{example}

\begin{example}
Dois navios começam a se deslocar a partir do mesmo ponto no instante
$t = 0$. Um navio se desloca para o norte a $30\si{km/h}$, enquanto que
o outro navio se desloca para o leste a $40\si{km/h}$. Encontre a
distância entre os dois navios em função do tempo.
\begin{stepanalysis}
\item Os navios começam na origem; o eixo dos $y$ aponta para o norte;
o eixo dos $x$ aponta para o leste. O diagrama é mostrado na
Figura~\ref{fig:exships}. Os valores $x$ e $y$ representam as distâncias
entre a origem e o navio que se desloca para o leste e para o norte,
respectivamente e $z$ é a distância entre os navios, todos em quilômetros.
A variável $t$ é o tempo, medido em horas.
\includefig{exships}
\item $t \ge 0$, \hspace{1.5em} $y = 30t$, \hspace{1.5em} $x = 40t$,
      \hspace{1.5em} $z = \sqrt{x^2 + y^2}$.
\item $x = \sqrt{(30t)^2 + (40t)^2} = \sqrt{2500t^2} = 50t$.
\end{stepanalysis}
\begin{interpretsolution} $z = 50t$, \hspace{1.5em} $t \ge 0$.

Onde $t$ é uma variável independente, $x$, $y$ e $z$ dependem de $t$.
A distância entre os navios é de $50t$ quilômetros, onde $t$ é o
tempo em horas.
\end{interpretsolution}
\end{example}

\begin{example}
  Um incêndio florestal começa ao longo de um segmento de reta com
  $20\si{m}$ de comprimento e se expande em todas as direções à taxa
  de $2\si{m/s}$. Determine a área queimada em função do tempo.
  \begin{stepanalysis}
  \item A partir do diagrama na Figura~\ref{fig:exbrushfire}, temos:
        \begin{eqnarray*}
          A & = & \text{área total queimada} \\
          A_1 & = & \text{área do semicírculo esquerdo} \\
          A_2 & = & \text{área do retângulo central} \\
          A_3 & = & \text{área do semicírculo direito} \\
          s & = & \text{distância do fogo em expansão} \\
          t & = & \text{tempo}
        \end{eqnarray*}
  \includefig{exbrushfire}
  \item As relações entre as variáveis são:
        \begin{eqnarray*}
          s & = & 2t, \hspace{1.5em} t \ge 0 \\
          A_1 & = & \frac{1}{2}\pi s^2, \hspace{1.5em} A_2 = 20(2s),
            \hspace{1.5em} A_3 = \frac{1}{2} \pi s^2, \\
          A & = & A_1 + A_2 + A_3.
        \end{eqnarray*}
  \item Determinamos $A$ em função de $t$,
        \begin{eqnarray*}
          A_1 & = & \frac{1}{2}\pi(2t)^2 = 2 \pi t^2,\\
          A_2 & = & 20 \cdot 2 \cdot 2t = 80t, \\
          A_3 & = & \frac{1}{2}\pi(2t)^2 = 2\pi t^2, \\
          A & = & 2\pi t^2 + 80t + 2\pi t^2 = 4\pi t^2 + 80t. \\
        \end{eqnarray*}
  \end{stepanalysis}
  \begin{interpretsolution}
    A área queimada é de $A = 4\pi t^2 + 80t \si{km^2}$, onde $t \ge 0$
    é o tempo em segundos.
  \end{interpretsolution}
\end{example}

Uma identidade algébrica que aparece frequentemente em problemas de cálculo
é
$$
  (a - b)(a + b) = a^2 - b^2.
$$
Às vezes ela ocorre na forma
$$
  (\sqrt{a} - \sqrt{b})(\sqrt{a} + \sqrt{b}) = a - b.
$$

\begin{example}
A área do quadrado $A$ é doze unidades de área maior do que a área do
quadrado $B$, e o lado de $A$ é três unidades de comprimento maior do
que o lado de $B$. Encontre as áreas de $A$ e $B$.
\begin{stepanalysis}
\item Seja $a$ a área de $A$ e $b$ a área de $B$. Veja
      Figura~\ref{fig:exsquarearea}.
\includefig{exsquarearea}
\item Os lados dos quadrados possuem comprimento $\sqrt{a}$ e $\sqrt{b}$,
      respectivamente. Portanto,
      $$
        a - b = 12, \hspace{1.5em} \sqrt{a} - \sqrt{b} = 3.
      $$
\item Encontramos $\sqrt{a} + \sqrt{b}$.
      \begin{eqnarray*}
        (\sqrt{a} - \sqrt{b})(\sqrt{a} + \sqrt{b}) & = & a - b, \\
        3(\sqrt{a} + \sqrt{b}) & = & 12, \\
        \sqrt{a} + \sqrt{b} & = & 4.
      \end{eqnarray*}
      Somando as equações $\sqrt{a} + \sqrt{b} = 4$ e
      $\sqrt{a} - \sqrt{b} = 3$, obtemos $2\sqrt{a} = 7$, \;
      $\sqrt{a} = \frac{7}{2}$, \; $a = \frac{49}{4}$. A subtração das
      equações dá $2 \sqrt{b} = 1$, \; $\sqrt{b} = \frac{1}{2}$, \;
      $b = \frac{1}{4}$.
\end{stepanalysis}
\begin{interpretsolution}
  A área do quadrado $A$ é $\frac{49}{4}$ unidades de área e a área do
  quadrado $B$ é $\frac{1}{4}$ unidades de área.
\end{interpretsolution}
\end{example}

\begin{sectionproblems}
\exer{Encontre o perímetro $p$ de um quadrado em função de sua área $A$.}

\exer{Um pedaço de argila no formato de um cubo de lado $s$ é moldado
      até se tornar uma esfera de raio $r$. Encontre $r$ em função de $s$.}

\exer{Encontre o volume $V$ de uma esfera em função da área da sua
      superfície $S$.}

\exer{Encontre a área $A$ de um retângulo com perímetro $4$ em função do
      comprimento $x$ de um de seus lados.}

\exer{Encontre a distância $z$ entre a origem e um ponto na parábola
      $y = 1 - x^2$ em função de $x$.}

\exer{Exprima o perímetro $p$ de um triângulo retângulo em função de
      sua base $x$ e altura $y$.}

\exer{Quatro pequenos quadrados, cada um de lado $x$, são cortados dos
      cantos de um grande quadrado de cartolina cujo lado é $s$. Os
      lados do quadrado grande são dobrados para formar uma caixa sem
      tampa. Encontre o volume da caixa em função de $s$ e $x$.}

\exer{Uma escada de comprimento $L$ é apoiada contra uma parede com a
      sua parte inferior à distância $x$ da parede. Encontre a altura
      $y$ do topo da escada em função de $x$.}

\exer{Um homem de altura $y$ está a $3\si{m}$ de uma lâmpada posicionada
      a $10\si{m}$ de altura. Encontre o comprimento $s$ de sua sombra
      em função de $y$.}

\exer{Um navio está se deslocando para o norte a $30\si{\km/m}$ e passa
      pela origem no instante $t = 0$ hora. Um segundo navio, se deslocando
      para o leste a $30\si{km/h}$ passa pela origem em $t = 1$ hora.
      Encontre a distância $z$ entre eles em função de $t$.}

\exer{Uma bola é jogada do nível do solo e a sua trajetória segue as
      equações
      $$
        x = bt, \hspace{1.5em} y = t - 16t^2.
      $$
      Que distância a bola percorrerá na direção $x$ até que ela atinja
      o solo?}

\exer{Um campo de ervas daninhas inicialmente ocupa um espaço circular
      com $2\si{m}$ de raio. O seu crescimento é tal que o seu raio
      aumenta de $1\si{m}$ por dia. Encontre a sua área após cinco dias.}

\exer{Um retângulo tem, originalmente, comprimento $\ell$ e largura $w$.
      Sua forma é alterada de tal modo que seu comprimento aumenta de
      uma unidade por segundo, enquanto que sua largura decresce de $2$
      unidades por segundo. Encontre sua área em função de $\ell$, $w$
      e do tempo $t$.}

\exer{Para cada $p$ unidades de poluição por ítem, um produto pode ser
      feito ao custo de $2 + 1/p$ reais por ítem. É preciso produzir
      $x$ ítens, com poluição total de uma unidade de poluição. Encontre
      o custo.}

\exer{Em economia, o lucro de produção e venda de $x$ ítens é igual ao
      faturamento menos o custo,
      $$
        L(x) = F(x) - C(x),
      $$
      Se um produto pode ser fabricado a um custo de R\$10 por
      ítem e se $x$ ítens podem ser vendidos a um preço de $100-\sqrt{x}$
      por ítem, encontre o lucro em função de $x$.}

\exer{Suponha que a demanda por um produto vendido a preço $p$ é de
      $x = 1000/\sqrt{p}$, ou seja, $x = 1000/\sqrt{p}$ ítens podem ser
      vendidos quando o preço é de $p$ reais por ítem. Se o custo de
      produção de $x$ ítens é de $100+10x$ reais, encontre o lucro em
      função do preço de venda $p$.}
\end{sectionproblems}

\section{Taxas Relacionadas}
\label{sec:relatedrates}

Em um problema de taxas relacionadas, a taxa de variação de uma quantidade
nos é fornecida, e desejamos encontrar a taxa de variação de outra. Tais
problemas frequentemente podem ser resolvidos por diferenciação implícita.

\begin{example}
  A ponta de uma caneta tinteiro é colocada sobre um papel absorvente,
  formando um círculo de tinta cuja área aumenta à razão constante de
  $0,03\si{cm^2/s}$. Encontre a taxa de variação do raio do círculo quando
  o círculo tem raio de $\frac{1}{2}\si{cm}$. Resolveremos o problema em
  $4$ etapas.
\begin{stepanalysis}
\item Rotule todas as quantidades envolvidas e desenhe um diagrama.
      $$
        t = \text{tempo}, \hspace{1.5em} A = \text{área}, \hspace{1.5em}
        r = \text{raio do círculo}
      $$
      Tanto $A$ quanto $r$ são funções de $t$. O diagrama é mostrado na
      Figura~\ref{fig:inkblot}.
\includefig{inkblot}
\item Escreva a informação dada na forma de equações.
      $$
        \dif A/\dt = 0,03, \hspace{1.5em} A = \pi r^2.
      $$
      O problema se resume a encontrar $\dif r/\dt$ quando $r = 1/2$.
\item Derive ambos os lados da equação $A = \pi r^2$ com relação a $t$.
      $$
        \dod A t = 2\pi r \dod r t, \text{ portanto } 0,03 = 2 \pi r
                                                      \dod r t .
      $$
\item Fixe $r = \frac{1}{2}$ e resolva para $\dif r/\dt$.
      $$
        0,03 = 2 \pi \frac{1}{2} \cdot \dod r t, \text{ portanto }
          \dod r t = \frac{0,03}{\pi}\si{cm/s}.
      $$
\end{stepanalysis}
\end{example}

\begin{example}
Uma escada de $3\si{m}$ de altura está apoiada contra uma parede. A parte
inferior da escada é afastada da parede à velocidade constante de
$0,6\si{m/s}$. Encontre a velocidade com que o topo da escada está
escorregando para baixo ao longo da parede quando a parte inferior está
a $1,5\si{m}$ da parede. \emph{Atenção:} apesar de a parte inferior da
escada estar sendo deslocada a uma velocidade constante, a velocidade de
deslocamento da parte superior varia com o tempo.
\begin{stepanalysis}
\item $t$ = tempo,\\
      $x$ = distância entre a parede e a parte inferior,\\
      $y$ = altura entre o topo da escada e o chão.

      O diagama é mostrado na Figura~\ref{fig:slidingladder}.
\includefig{slidingladder}
\item $\dx/\dt = 0,6$, \hspace{1.5em} $x^2 + y^2 = 3^2 = 9$.
\item Derivamos ambos os lados de $x^2 + y^2 = 9$ com relação a $t$.
      $$
        2x \dod x t + 2y \dod y t = 0, \text{ portanto }
          1,2 x + 2y \dod y t = 0.
      $$
\item Fixe $x = 1,5\si{m}$ e resolva para $\dy/\dt$. Encontramos
      primeiramente o valor de $y$ quando $x = 1,5$.
      $$
        x^2 + y^2 = 9, \hspace{1.5em} y = \sqrt{9 - x^2} =
          \sqrt{9 - 1,5^2} = \sqrt{6,75}.
      $$
      Então, resolvemos para $\dy/\dt$.
      \begin{eqnarray*}
        1,2 x + 2 y \dod y t & = & 0, \\
        1,2 \cdot 1,5 + 2 \sqrt{6,75} \dod y t & = & 0, \\
        \dod y t & = & -\frac{1,2 \cdot 1,5}{2 \sqrt{6,75}} =
          - \frac{1,8}{2 \sqrt{6,75}} \sim 0,35\si{m/s}.
      \end{eqnarray*}
      O sinal de $\dy/\dt$ é negativo pois $y$ está diminuindo.
\end{stepanalysis}
\end{example}

Problemas de taxas relacionadas geralmente assumem a forma a seguir.

\emph{Dados:}
\begin{enumerate}[(a)]
\item Duas quantidades que dependem do tempo, a saber, $x$ e $y$.
\item A taxa de variação de uma delas, por exemplo, $\dx/\dt$.
\item Uma equação que mostra a relação entre $x$ e $y$.
\end{enumerate}
(Geralmente, esta informação é dada na forma de uma descrição verbal de
uma situação física e parte do problema é exprimi-lo na forma de uma
equação.)

\emph{O problema:} encontrar a taxa de veriação de $y$ com relação ao
tempo, $\dy/\dt$, em um dado instante $t_0$. (Às vezes, o instante $t_0$
é especificado por algum valor dado para $x$ ou $y$ naquele instante.)

Problemas de taxas relacionadas frequentemente podem ser resolvidos em
quatro passos, conforme fizemos nos exemplos anteriores.
\begin{stepanalysis}
\item Rotule todas as quantidades no problema e construa um diagrama.
      Considerando que os rótulos sejam $x$, $y$ e $t$ (tempo), os
      passos restantes são os seguintes:
\item Escreva uma equação para a taxa de variação $\dx/\dt$, a qual
      será dada no enunciado. Escreva outra
      equação para a relação dada entre $x$ e $y$.
\item Derive ambos os lados da equação que relaciona $x$ e $y$ com
      relação a $t$. Escolhemos o tempo $t$ como a variável independente.
      O resultado é uma nova equação envolvendo $x$, $y$, $\dx/\dt$ e
      $\dy/\dt$.
\item Fixe $t = t_0$ e resolva para $\dy/\dt$. Pode ser necessário
      determinar primeiramente os valores de $x$, $y$ e $\dx/\dt$ no
      instante $t = t_0$.
\end{stepanalysis}

O passo mais complexo é geralmente o segundo, já que é necessário
interpretar a descrição verbal do problema para obtermos um sistema
de fórmulas. Algumas vezes, são dadas mais de duas quantidades que
dependem do tempo. O próximo exemplo tem três destas quantidads.

\begin{example}
Um carro se desloca para o norte a $40\si{km/h}$ e passa um ponto $P$
à $1$ hora. Outro carro se move para o leste a $60\si{km/h}$ e
passa pelo mesmo ponto $P$ às $2\si{h}30\si{min}$. Qual é a variação
da distância entre os dois carros às $2\si{h}$?

Note que o enunciado sequer informa se os carros estão se
aproximando ou se afastando às $2\si{h}$. Isto é parte do problema.
\begin{stepanalysis}
\item $t$ = tempo,\\
      $y$ = posição do carro que trafega para o norte,\\
      $x$ = posição do carro que trafega para o leste,\\
      $z$ = distãncia entre os dois carros.

      No diagrama da Figura~\ref{fig:carsnortheast}, o ponto $P$ é
      colocado na origem.
\includefig{carsnortheast}
\item $\dsfrac{\dy}{\dt} = 40$, \hspace{1.5em}
      $\dsfrac{\dx}{\dt} = 60$, \hspace{1.5em}
      $x^2 + y^2 = z^2$.
\item $2x\dsfrac{\dx}{\dt} + 2y\dsfrac{\dy}{\dt} = 2z\dsfrac{\dz}{\dt}$,
      \;\;\; portanto \;\;\; $60x + 40y = z\dsfrac{\dz}{\dt}$
\item Primeiramente, encontramos os valores de $x$, $y$ e $z$ no instante
      $t = 2\si{h}$. Temos que em $t = 1$, $y = 0$. Na próxima hora, o
      carro se desloca $40\si{km}$, então em $t = 2$, $y = 40$. Temos também
      que no instante $t = 2,5$, $x = 0$. Meia hora antes disto, o carro
      estava $30\si{km}$ à esquerda de $P$, então em $t = 2$, $x = -30$.
      Em suma:
      $$
        \text{em } t = 2, \hspace{1.5em} y = 40 \hspace{1.5em} \text{e}
          \hspace{1.5em} x = -30.
      $$
      Agora, podemos encontrar o valor de $z$ em $t = 2$,
      $$
        z = \sqrt{x^2 + y^2} = \sqrt{(-30)^2 + 40^2} = 50.
      $$
      Finalmente, podemos resolver para $\dz/\dt$ em $t = 2$,
      $$
        60 \cdot (-30) + 40 \cdot 40 = 50 \dod z t, \hspace{1.5em}
          \dod z t = \frac{-1800 + 1600}{50} = -4.
      $$
      O sinal negativo indica que $z$ está diminuindo. Assim, às
      $2\si{h}$, os carros estão se aproximando a uma velociade
      relativa de $4\si{km/h}$ entre eles.
\end{stepanalysis}
\end{example}

\section{Limites}
\label{sec:limits}

\section{Continuidade}
\label{sec:continuity}

\section{Máximos e Mínimos}
\label{sec:maxmin}

\section{Máximos e Mínimos -- Aplicações}
\label{sec:maxminappl}

\section{Derivadas e Esboços de Curvas}
\label{sec:derivsketch}

\section{Propriedades de Funções Contínuas}
\label{sec:propcont}

\begin{chapterproblems}
\end{chapterproblems}


