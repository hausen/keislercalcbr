\chapter{Demonstrações Adicionais}

As demonstrações a seguir não estão na versão original deste texto,
e foram incluídas pelo tradutor por se tratarem de propriedades
importantes vistas na abordagem tradicional do cálculo.

\section{Teorema do Confronto}
\index{teorema!do confronto}

\begin{theorem}
Sejam $(a,b)$ um intervalo aberto não vazio tal que $x_0 \in (a,b)$, $L$ um
número real e $f,g,h$ funções reais que satisfazem as propriedades
abaixo:
\begin{itemize}
\item $g(x) \le f(x) \le h(x)$ para todo $x \in (a,b) \setminus \{ x_0 \}$
\item $\lim_{x \rightarrow x_0} g(x) = \lim_{x \rightarrow x_0} h(x) = L.$
\end{itemize}
Então $\lim_{x \rightarrow x_0} f(x) = L$.
\end{theorem}

\begin{proof}
Seja $\hyper{x}$ um número hiper-real tal que $\hyper{x} \approx x_0$ mas
$\hyper{x} \ne x_0$. Como $a < x_0 < b$, tal número existe e
$a < \hyper{x} < b$. Considere as
extensões naturais de $f,g,h$ aos hiper-reais, denotadas respectivamente
por $\hyper{f}$, $\hyper{g}$ e $\hyper{h}$.

Como vale $g(x) \le f(x) \le h(x)$ para todo real $x \ne x_0$ tal que
$a < x < b$, pelo Princípio da Extensão também vale
$\hyper{g}(\hyper{x}) \le \hyper{f}(\hyper{x}) \le \hyper{h}(\hyper{x})$.

Consequentemente, 
$0 \le \hyper{f}(\hyper{x}) - \hyper{g}(\hyper{x}) \le \hyper{h}(\hyper{x}) - \hyper{g}(\hyper{x})$. Como $\hyper{g}(\hyper{x}) \approx L$ e $\hyper{h}(\hyper{x}) \approx L$, então existem infinitésimos $\epsilon_1, \epsilon_2$ tais que
$\hyper{g}(\hyper{x}) = L + \epsilon_1$ e $\hyper{h}(\hyper{x}) = L + \epsilon_2$. Seja $\epsilon = \epsilon_1 - \epsilon_2$ um infinitésimo. Temos que $0 \le \hyper{f}(\hyper{x}) - \hyper{g}(\hyper{x}) \le \epsilon < r$ para todo número real $r > 0$. Isto implica que a diferença $\hyper{f}(\hyper{x}) - \hyper{g}(\hyper{x})$ é
infinitesimal, portanto $\hyper{f}(\hyper{x}) \approx \hyper{g}(\hyper{x})$, logo $\hyper{f}(\hyper{x}) \approx L$. Disto decorre que $\lim_{x \rightarrow x_0} f(x) = L$.
\end{proof}

