\graphicspath{ {./figuras/98_epilogue/} }
\message{READING 98_epilogue.tex}
\chapter*{Epílogo}
\addcontentsline{toc}{chapter}{Epílogo}

\setcounter{chapter}{5}% Equivalent to "letter E"
\renewcommand{\thechapter}{\Alph{chapter}}%
\renewcommand{\thefigure}{\thechapter.\arabic{figure}}

\markboth{Epílogo}{Epílogo}

Como o cálculo por infinitésimos, da maneira como desenvolvido neste livro,
se relaciona com a abordagem tradicional para o cálculo, por meio de
épsilons e deltas ($\epsilon,\delta$)?
Para vermos as coisas pela perspectiva correta,
esboçaremos a história do cálculo.

Muitos problemas envolvendo inclinações de retas, áreas e volumes, os quais
chamaríamos hoje de problemas de cálculo, foram resolvidos por antigos
matemáticos gregos. O maior de todos foi Arquimedes (287--212 A.C.).
Arquimedes antecipou tanto a abordagem por infinitésimos quanto a
abordagem por $\epsilon,\delta$. Por vezes, ele descobria seus resultados
através do raciocínio por infinitésimos, mas sempre publicava suas provas
usando o ``método da exaustão,'' que é similar à abordagem moderna por
$\epsilon,\delta$.

Problemas de cálculo tornaram-se importantes no início do século
XVII com o desenvolvimento da física e da astronomia. As regras
básicas de derivação e integração foram descobertas naquele período
por meio do raciocínio informal com infinitésimos. Kepler, Galileu,
Fermat e Barrow estiveram entre aqueles que contribuíram para esse
desenvolvimento.

Entre os anos de 1660 até o final da década de 1670, Sir Isaac Newton e
Gottfried Wilhelm Leibnitz ``inventaram'' independentemente o cálculo.
Eles deram o maior passo de reconhecer a importância de uma coleção
de resultados isolados e organizá-los em um todo.

Newton, em várias épocas, descreveu a derivada de $y$ (a qual ele
denominou ``fluxão\tnote{do original, ``fluxion.''}'' de $y$) de
três maneiras diferentes, em linhas gerais
\begin{enumerate}[(1)]
\item A razão entre uma mudança infinitesimal em $y$ em relação a
      uma mudança infinitesimal em $x$. (O método por infinitésimos.)
\item O limite da taxa de mudança em $y$ com relação à mudança em $x$,
      $\Delta y / \Delta x$, à medida que $\Delta x$ se aproxima de
      zero. (O método pelo limite.)
\item A velocidade de $y$ onde $x$ denota o tempo. (O método pela
      velocidade.)
\end{enumerate}

Em seus últimos escritos, Newton procurou evitar infinitesimais e
enfatizou os métodos (2) e (3). 

Leibiniz, por sua vez, consistentemente favorecia o método por infinitésimos,
mas acreditava (corretamente) que os mesmos resultados poderiam ser
obtidos usando-se apenas números reais. Ele considerava os infinitésimos
como números ``ideais,'' como os números imaginários. Para justificá-los,
ele propôs a lei da continuidade: ``Em qualquer transição suposta,
terminando em qualquer conclusão, é permissível instituir uma argumentação
geral, na qual a conclusão pode também ser incluída.''\footnote{Veja Kline, p. 385 e Boyer, p. 217} Esta ``lei'' é imprecisa demais para os padrões
atuais. Mas foi uma precursora notável do Princípio de Transferência
no qual o cálculo por infinitésimos moderno é baseado. Leibiniz estava
no caminho correto, mas chegou a ele 300 anos por demais cedo!

A notação desenvolvida por Leibinitz ainda está em uso geral nos dias de hoje,
mesmo tendo sido criada para sugerir o uso do método por infinitésimos:
$dy/dx$ para a derivada (para sugerir uma mudança infinitesimal em $y$
dividida por uma mudança infinitesimal em $x$), e $\int_b^a f(x) \mathrm{dx}$ para a integral (para sugerir a soma de uma quantidade infinita de
infinitésimos $f(x) \mathrm{dx}$).

Todas as três abordagens tinham sérias inconsistências, as quais foram
criticadas mais efetivamente por George Berkley\tnote{no original, ele
é denominado ``Bispo Berkley''} em 1734. Entretanto, um tratamento
preciso do cálculo estava além do estado da arte à época, e as
três descrições intuitivas da derivada (1)--(3) competiram entre si
pelos próximos duzentos anos. Até alguns anos após 1820,
o método infinitesimal (1) de Leibniz era o dominante no continente
europeu, por causa do seu apelo intuitivo e pela conveniência da
notação de Leibniz. Na Inglaterra, o método da velocidade (3)
predominava; ele também possui apelo intuitivo, mas não tem como
ser feito de maneira rigorosa.

Em 1821, A. L. Cauchy publicou um precursor do moderno tratamento para
o cálculo baseado no método pelo limite (2). Ele definiu tanto a integral
quanto a derivada em termos de limites, a saber
\[
\int_a^b f(x) \mathrm{dx} = \lim_{\Delta x \rightarrow 0^+} \sum_{a}^{b} f(x) \Delta x.
\]
Ele ainda usava infinitésimos, considerando-os como variáveis que
se aproximavam de zero. Daquele momento em diante, o método pelo
limite gradualmente se torno a abordagem predominante ao cálculom
enquanto que infinitésimos e apelos à velocidade sobreviveram apenas
como um modo de falar. Contudo, ainda havia dois pontos importantes a serem
tratados no trabalho de Cauchy. Primeiro, a definição de limite dada por
Cauchy não era suficientemente clasa; ainda dependia do uso intuitivo
de infinitésimos. Segundo, uma definição precisa do sistema de números
reais ainda não estava disponível. Tal definição precisaria de uma
melhor compreensão dos conceitos de conjunto e função, os quais ainda
estavam evoluindo.

Um tratamento completamente rigoroso do cálculo foi formulado finalmente
por Karl Weierstrass nos anos de 1870. Ele introduziu a condição
$\epsilon,\delta$ como a definição de limite. Aproximadamente nessa
época, alguns matemáticos (incluindo Weierstrass), foram bem sucedidos
na construção dos números reais a partir dos inteiros positivos. O
problema de construção do sistema de números reais também levou ao
desenvolvimento da teoria de conjuntos por Georg Cantor na mesma
década. A abordagem de Weierstrass tornou-se o tratamento tradicional,
ou ``padrão,'' para o cálculo como ele é apresentado hoje. Tudo começa
com a condição $\epsilon,\delta$ como a definição de limite e continua
com o desenvolvimento do cálculo apenas em termos do sistema de
números reais (sem mensão a infinitésimos). Entretanto, mesmo quando
o cálculo é apresentado na forma padrão, costuma-se argumentar
informalmente em termos de infinitésimos, e usa-se a notação de
Leibniz que sugere infinitésimos.

Desde os tempos de Weierstrass até muito recentemente, aparentava-se
que o método pelo limite (2) havia finalmente ganho a disputa e
que a história do cálculo elementar estava encerrada. Mas em 1934,
Thoralf Skolem construiu o que denominamos aqui por hiperinteiros,
e demonstrou que o análogo do Princípio da Transferência é válido para
eles. A construção de Skolem (agora chamada de construção pelo
ultraproduto) foi posteriormente estendida a uma grande classe de
estruturas, incluindo a construção dos números hiper-reais a partir
dos números reais. O nome ``hiper-real'' foi primeiramente usado por
E. Hewitt em um artigo de 1948. Os números hiper-reais eram conhecidos
mais de uma década antes de serem aplicados ao cálculo.

Finalmente, em 1960, Abraham Robinson descobriu que os números hiper-reais
poderiam ser usados para dar um tratamento rigoroso ao cálculo com
infinitésimos. A apresentação do cálculo dada neste livro é baseada
no tratamento de Robinson (mas modificada para torná-la adequada a um
primeiro curso).

O cálculo de Robinson segue o espírito do antigo método por infinitésimos
de Leibniz. Há grandes diferenças nos detalhes. Por exemplo, Leibniz
definiu a derivada como a razão $\Delta y / \Delta x$ onde $\Delta x$
é infinitesimal, enquanto que Robinson define a derivada como sendo
a \emph{parte padronizada} da razão $\Delta y / \Delta x$ onde $\Delta x$
é infinitesimal. Esta é a forma como Robinson evita as inconsistências
na abordagem antiga por infinitésimos. Além disso, a vaga lei da
continuidade de Leibniz é substituída pela formulação precisa do
Princípio da Transferência.

A razão pela qual o trabalho de Robinson não ter sido realizado antes é
que o Princípio da Transferência para números hiper-reais é um tipo de
axioms que não era familiar na matemática até recentemente. Ele emergiu
% TODO: verificar se model theory = teoria dos modelos
no campo da teoria dos modelos, que estuda a relação entre axiomas e
estruturas matemáticas. Os desenvolvimentos pioneiros na teoria dos
modelos não tinham sido feitos até os anos de 1930, por Gödel, Malcev,
Skolem e Tarski; o assunto mal existia até a década de 1950.

Olhando para trás, vemos que o método por infinitésimos era geralmente
o preferido, com relação ao método pelo limite, durante mais de 150 após
Newton e Leibniz terem inventado o cálculo, pois infinitésimos possuem
grande apelo intuitivo. Mas o método pelo limite foi finalmente adotado
ao redor de 1870 pois foi o primeiro tratamento matemático preciso para
o cálculo. Hoje em dia também é possível usar infinitésimos de uma
maneira matematicamente precisa. Os infinitésimos, no sentido proposto
por Robinson, tem sido aplicados não somente ao cálculo, mas também
ao campo mais amplo da análise. Eles levaram a novos resultados e
problemas em pesquisa matemática. Como os hiperinteiros de Skolem são
usualmente chamados de ``inteiros não padronizados,'' Robinson chamou
o novo campo de ``análise não padronizada.'' (Ele chamou os números
reais de ``padronizados'' e os demais números hiper-reais de
``não padronizados.'' Esta é a origem do nome ``parte padronizada.'')

O ponto de partida para este curso foi um par de descrições intuitivas
dos sistemas de números reais e hiper-reais. Essas descrições são,
na verdade, apenas esboços rudimentares que não são completamente
fiáveis. Para nos assegurarmos que os resultados estão corretos, o
cálculo deve ser baseado em descrições matematicamente precisas desses
sistemas de números, que preenchem as frestas nas descrições intuitivas.
Há dois modos de se fazê-lo. A maneira mais rápida é listar as propriedades
matemáticas dos números reais e hiper-reais. Estas propriedades devem ser
aceitas como básicas, e são chamadas \emph{axiomas}. A segunda maneira
de descrever matematicamente os números reais e hiper-reais é começar com
os inteiros positivos e, passo a passo, construir os inteiros, os 
números racionais, os números reais, e os números hiper-reais. Este
segundo método é melhor pois mostra que há realmente um estrutura com as
propriedades desejadas. No final deste epílogo, delinearemos brevemente a
construção dos números reais e hiper-reais e daremos alguns exemplos de
infinitésimos.

Nos dedicaremos, neste momento, à primeira maneira de descrever
matematicamente os números reais e hiper-reais. Listaremos dois grupos
de axiomas neste epílogo, um para números reais, e um para números
hiper-reais. Os axiomas para os números hiper-reais serão apenas
afirmações mais cuidadosas do Princípio da Extensão e do Princípio
da Transferência do Capítulo~\ref{chp:reals}. Os axiomas para os
números reais são apresentados em três partes: os Axiomas Algébricos, os
Axiomas de Ordem, e o Axioma da Completude. Todos os fatos familiares
sobre os números reais podem ser provados usando-se apenas estes
axiomas.

\subpart{I. Axiomas Algébricos dos Números Reais}

\index{axioma!algébrico para os reais}
\begin{enumerate}[A]
\item Leis de fecho: $0$ e $1$ são números reais. Se $a$ e $b$ são
      números reais, também serão $a+b$, $ab$ e $-a$. Se $a$ é um
      número real e $a \ne 0$, então $1/a$ é um número real.
\item Leis de comutatividade: $a+b = b + a, \; \; ab = ba$
\item Leis de associatividade: $a+(b+c) = (a+b) + c, \; \; a(bc) = (ab)c$
\item Leis de identidade: $0 + a = a, \; \; 1 \cdot a = a.$
\item Leis de inversos: $a+(-a) = 0, \; \;$ Se $a \ne 0$, $a \cdot \frac{1}{a} = 1.$
\item Lei de distributividade: $a \cdot (b+c) = ab + ac.$
\end{enumerate}

\begin{defin}
Os \newdef{inteiros positivos} são os números reais $1$, $2 = 1+1$,
$3 = 1+1+1$, $4 = 1+1+1+1$ e daí em diante.
\end{defin}

\subpart{II. Axiomas de Ordem para Números Reais}

\index{axioma!de ordem para os reais}
\begin{enumerate}[A]
\item $0 < 1$
\item Lei da transitividade: Se $a < b$ e $b < c$ então $a < c$.
\item Lei da tricotomia: Exatamente uma das relações $a < b$, $a = b$ ou $a > b$ é válida.
\item Lei da soma: Se $a < b$, então $a + c < b + c$.
\item Lei do produto: Se $a < b$ e $0 < c$, então $ac < bc$.
\item Axioma da raiz: Para todo número real $a > 0$ e todo inteiro positivo
      $m$, há um número real $b > 0$ tal que $b^n = a$.
\end{enumerate}

\subpart{III. Axioma da Completude}

\index{axioma!da completude}
\emph{
Seja $A$ um conjunto de números reais tal que, toda vez que $x$ e $y$
estão em $A$, qualquer número real entre $x$ e $y$ também estará em $A$.
Então $A$ é um intervalo.
}

\begin{theor}
Uma sequência crescente $\langle S_n \rangle$ ou converge, ou diverge para
$\infty$.
\end{theor}

\begin{proof}
Seja $T$ o conjunto de todos os números reais $x$ tais que $x \le S_n$ para
algum $n$. Este conjunto é claramente não-vazio.

\begin{caseanalysis}
\item $T$ é toda a reta real. Se $H$ é infinito, temos que $x \le S_H$
para todo e qualquer número real $x$. Então $S_H$ é infinito positivo e
$\langle S_n \rangle$ diverge para $\infty$.

\item $T$ não é toda a reta real. Pelo Axioma de Completude,
$T$ é um intervalo $(-\infty,b]$ ou $(-\infty,b)$. Para todo número
real $x$ tal que $x < b$, temos
\[
	x \le S_n \le S_{n+1} \le S_{n+2} \le \ldots \le b
\]
para algum $n$. Disto decorre que, para um $H$ infinito, $S_H \le b$ e
$S_H \approx b$. Logo, $\langle S_n \rangle$ converge para $b$.
\end{caseanalysis}%
\end{proof}

A partir de agora, atacaremos o segundo grupo de axiomas, os quais
fornecem as propriedades dos números hiper-reais. Haverá dois axiomas,
chamados Axioma da Extensão e Axioma da Transferência, que correspondem
ao Princípio da Extensão e ao Princípio da Transferência da Seção%
~\ref{sec:infnumbers}. Enunciaremos primeiro o Axioma da Extensão.

\subpart{I*. Axioma da Extensão}

\index{axioma!da extensão}
\emph{
\begin{enumerate}[(a)]
\item O conjunto $\setR$ dos números reais é um subconjunto do conjunto
      $\setHR$ dos números hiper-reais.
\item Existe uma relação $\lth$ em $\setHR$, tal que:
      \begin{itemize}
      \item a relação de ordem $<$ em $\setR$ é a restrição de $\lth$
            aos reais;
      \item $\lth$ é transitiva ($a \lth b$ e $b \lth c$ implica $a \lth c)$; e
      \item $\lth$ satisfaz a Lei da Tricotomia (para quaisquer $a,b$ em $\setHR$, exatamente uma das afirmações $a \lth b$, $a = b$ ou $b \lth a$ é verdadeira).
      \end{itemize}
\item Existe um número hiper-real $\epsilon$ tal que $0 \lth \epsilon$ e
      $\epsilon \lth r$ para todo número real positivo $r$.
\item Para cada função real $f$, existe uma função hiperreal $\hyper{f}$
      com o mesmo número de variáveis, chamada \emph{extensão natural}
      de $f$ aos hiper-reais.
\end{enumerate}
}

A parte (c) do Axioma da Extensão afirma que há pelo menos um infinitésimo
positivo. A parte (d) nos dá a extensão natural para cada função real.
O Axioma da Transferência nos dirá que esta extensão natural possui as
mesmas propriedades da função original.

Lembre que o Princípio da Transferência da Seção~\ref{sec:infnumbers} fez
o uso da ideia intuitiva de uma proposição sobre os reais. Antes de podermos
enunciar o Axioma da Transferência, devemos dar uma explicação matematicamente
precisa da noção de uma proposição sobre os reais. Isto será feito em
vários passos, primeiro introduzindo-se os conceitos de uma expressão
real e de uma fórmula.

Começamos com o conceito de uma \newdef{expressão} real, ou \newdef{termo},
construído a partir de variáveis e constantes reais, usando funções reais.
Expressões reais podem ser construídas da seguinte forma:
\begin{enumerate}[(1)]
\item Uma constante real sozinha é uma expressão real.
\item Uma variável real sozinha é uma expressão real.
\item Se $e$ é uma expressão real, e $f$ é uma função real de uma variável,
      então $f(e)$ é uma expressão real. Similarmente, se $e_1, \ldots, e_n$
      são expressões reais e $g$ é uma função real de $n$ variáveis, então
      $g(e_1, \ldots, e_n)$ é uma expressão real.
\end{enumerate}

O passo (3) pode ser usado repetidamente para se construir expressões
mais longas. Alguns exemplos de expressões reais, onde $x$ e $y$ são
variáveis:
\[
2, \hspace{3ex} x+y, \hspace{3ex} |x-4|, \hspace{3ex} \sen(\pi y^2),
\hspace{3ex} \frac{\sqrt{x} + \sqrt{y}}{\sqrt{3}}, \hspace{3ex} g(x,f(0)),
\hspace{3ex} 1/0.
\]

Por \newdef{fórmula} queremos dizer uma afirmação de um dos tipos a seguir,
onde $d$, $e$ são expressões reais:
\begin{enumerate}[(1)]
\item uma \newdef{equação} entre duas expressões reais, $d = e$.
\item uma \newdef{inequação} entre duas expressões reais, $d < e$ ou
      $d \le e$ ou $d > e$ ou $d \ge e$ ou $d \ne e$.
\item uma afirmação na forma ``$e$ é definido'' ou ``$e$ é indefinido.''
\end{enumerate}
Eis aqui alguns exemplos de fórmulas:
\begin{eqnarray*}
  x + y & = & 5, \\
  f(x)  & = & \frac{1 - x^2}{1 + x}, \\
 g(x,y) & < & f(t), \\
 f(x,x) & \text{é} & \text{indefinido}.
\end{eqnarray*}
Se cada variável em uma fórmula for substituída por um número real, a
fórmula será ou verdadeira, ou falsa. Ordinariamente, a fórumula será
verdadeira para alguns valores de variáveis e falsa para outros. Por
exemplo, a fórmula $x+y = 5$ será verdadeira quando $(x,y) = (4,1)$
e falsa quando $(x,y) = (7,-2)$.

\begin{defin}
Uma \newdef[proposição!sobre os reais]{proposição sobre os reais}
é ou um conjunto finito e não vazio
de fórmulas $T$, ou uma combinação envolvendo dois conjuntos finitos e
não vazios de fórmulas $S$ e $T$ que afirma que ``sempre que todas as
fórmulas em $S$ forem verdadeiras, então todas as fórmulas em $T$ serão
verdadeiras.''
\end{defin}

Daremos vários comentários e exemplos para esclarecer esta definição. Por
vezes, ao invés de escrevermos ``sempre que todas as
fórmulas em $S$ forem verdadeiras, então todas as fórmulas em $T$ serão
verdadeiras'' usaremos a forma mais sucinta ``se $S$ então $T$'' para
uma proposição sobre os reais. Cada um dos Axiomas Algébricos para os Números
Reais é uma proposição sobre os reais. As leis da comutatividade,
associatividade, identidade e distributividade são proposições sobre os
reais. Por exemplo, as leis da comutatividade são um par de fórmulas
\[
a  + b = b + a, \hspace{3ex} ab = ba,
\]
que envolvem duas variáveis $a$ e $b$. As leis de fecho podem ser expressas
como quatro proposições sobre os reais:
\begin{eqnarray*}
a + b & \text{é} & \text{definido}, \\
ab    & \text{é} & \text{definido}, \\
-a    & \text{é} & \text{definido}, \\
\text{se } a \ne 0 \text{ então } 1/a & \text{é} & \text{definido}.
\end{eqnarray*}
As regras de inversos consistem de mais duas proposições sobre os reais. A
Lei da Tricotomia é parte do Axioma da Extensão, e todos os outros
Axiomas de Ordem para os Números Reais são proposições sobre os reais.
Contudo, o Axioma da Completude não é uma proposição sobre os reais,
pois não é construído a partir de equações e desigualdades entre
termos.

Um exemplo típico de proposição sobre os reais é a inequação para expoentes
discutida na Seção~\ref{sec:expfunc}:
$$
  \text{se } a \ge 0 \text{ e } q \ge 1\text{, então } (a+1)^q \ge aq + 1.
$$
Esta proposição é verdadeira para todos os números reais $a$, $q$.

Um significado para essa fórmula pode ser dado tanto no sistema de números
hiper-reais quanto no sistema de números reais. Considere uma fórmula com
as duas variáveis $x$ e $y$. Quando $x$ e $y$ são substituídos por números
reais particulares, a fórmula ou será verdadeira, ou será falsa no sistema de
números reais. Para dar um significado à fórmula no sistema de números
hiper-reais, substituimos cada função real pela sua extenção e substituimos
a relação de ordem real $<$ pela relação hiper-real $\hyper{<}$. Quando
$x$ e $y$ são substituídos por números reais, cada função real $f$ é
substituída por sua extensão natural $\hyper{f}$, e a relação de ordem
real $<$ é substituída por $\hyper{<}$, a fórmula ou será verdadeira, ou
será falsa no sistema de números hiper-reais.

Por exemplo, a fórmula $x + y = 5$ é verdadeira no sistema de números
hiper-reais quando $(x,y) = (2 - \epsilon, 3 + \epsilon)$, mas falsa
quando $(x,y) = (2+\epsilon, 3 + \epsilon)$, se $\epsilon$ não é nulo.

Estamos prontos para enunciar o Axioma da Transferência.

\subpart{II*. Axioma da Transferência}

\index{axioma!da transferência}
\emph{Cada proposição real que é aplicável para todos os números reais
também é aplicável para todos os números hiper-reais.}

É possível desenvolver todo o curso de cálculo como apresentado neste
livro a partir destes axiomas para os números reais e hiper-reais. Pelo
Axioma de Transferência, todos os Axiomas Algébricos para os Números
Reais também são verdadeiros para os números hiper-reais. Podemos
transferir todos os Axiomas de Ordem dos números reais para os números
hiper-reais. A Lei de Tricotomia é parte do Axioma da Extensão. Cada um
dos demais Axiomas de Ordem é uma proposição real e, como tal, é
transponível aos números hiper-reais pelo Axioma da Transferência. Desta
forma, podemos computar com os números hiper-reais da mesma forma que
fazemos para os números reais.

Há um fato de fundamental importância que será enunciado a seguir como
um teorema.

\begin{theor}[Princípio da Parte Padronizada]
\index{princípio!da Parte Padronizada}
Para todo número hiper-real finito $b$, existe exatamente um único
número real $r$ que está infinitamente próximo de $b$.
\end{theor}

\begin{proof}
Primeiramente, demonstraremos que não pode haver mais de um número real
infinitamente próximo de $b$. Suponha que $r$ e $s$ são números reais
tais que $r \approx b$ and $s \approx b$. Então, $r \approx s$ e como
$r$ e $s$ são reais, $r$ deve ser igual a $s$. Portanto, há no máximo
um número real infinitamente próximo de $b$.

Demonstraremos agora que há um número real infinitamente próximo de $b$.
Seja $A$ o conjunto de todos os números reais menores do que $b$. Então,
todo número real entre dois elementos de $A$ é um elemento de $A$. Pelo
Axioma da Completude para os números reais, $A$ é um intervalo. Como o
número hiper-real $b$ é finito, $A$ deve ser um intervalo da forma
$(-\infty, r)$ ou $(-\infty, r]$ para algum número real $r$. Todo número
real $s < r$ é um elemento de $A$, portanto $s < b$. Além disso, todo
número real $t > r$ não pertence a $A$, logo $t \ge b$. Isto demonstra
que $r$ está infinitamente próximo de $b$.
\end{proof}

Frisamos anteriormente que o Axioma da Completude não se qualifica como
uma proposição real. Por esta razão, o Princípio da Transferência não
pode ser usado para transferir o Axioma da Completude aos números
hiper-reais. Na verdade, o Axioma da Completude \emph{não é verdadeiro}
para os números hiper-reais. Dizemos que um \emph{intervalo hiper-real
fechado} é um conjunto de números hiper-reais na forma $[a,b]$, ou seja,
o conjunto de todos os números hiper-reais $x$ tais que $a \le x \le b$,
onde $a, b$ são constantes hiper-reais. Intervalos hiper-reais abertos
e semiabertos são definidos de maneira similar. Quando dizemos que o
Axioma da Completude não é válido para os números hiper-reais, queremos
dizer que efetivamente há conjuntos $A$ de números hiper-reais tais que:

\begin{enumerate}[(a)]
\item Toda vez que $x$ e $y$ pertencem a $A$, qualquer número hiper-real
entre $x$ e $y$ está em $A$.
\item $A$ não é um intervalo hiper-real.
\end{enumerate}

Eis aqui dois exemplos bem familiares.

\begin{example}
O conjunto $A$ de todos os infinitésimos possui a propriedade (a) acima
mas não é um intervalo hiper-real. Ele satisfaz a propriedade (a) pois
qualquer número hiper-real que está entre dois infinitésimos é também um
infinitésimo. Mostraremos que $A$ não é um intervalo hiper-real. Veja que
$A$ não pode ser da forma $[a,+\infty)$, nem $(a,+\infty)$, pois se $b$
é um infinitésimo positivo, então $2 \cdot b$ é um infinitésimo maior.
Por outro lado, $A$ não pode ser da forma $[a,b)$, nem $(a,b)$, pois se
$b$ é positivo mas não infinitesimal, então $b/2$ é menor do que $b$, mas
ainda assim positivo e não infinitesimal.
\end{example}

O conjunto $B$ de todos os números hiper-reais finitos é um outro exemplo
de um conjunto que possui a propriedade (a) acima, mas não é um intervalo.

Aqui temos alguns exemplos que podem ajudar a ilustrar a natureza do
sistema de números hiper-reais e o uso do Axioma de Transferência.

\begin{example}
Seja $f$ a função real dada pela equação
$$
  f(x) = \sqrt{1-x^2}.
$$
Seu gráfico é o semicírculo unitário com centro na origem. As duas proposições
reais a seguir são válidas para todo número real $x$:
\begin{eqnarray*}
  \text{sempre que } 1-x^2 \ge 0, & & f(x) = \sqrt{1-x^2}; \\
  \text{sempre que } 1-x^2 < 0,   & & f(x) \text{ está indefinido.} \\
\end{eqnarray*}
Pelo Axioma de Transferência, estas proposições reais também são válidas
para todo número hiper-real $x$. Portanto, a extensão natural de $\hyper{f}$
de $f$ é dada pela mesma equação
$$
  \hyper{f}(x) = \sqrt{1-x^2}.
$$
O domínio de $\hyper{f}$ é o conjunto de todos os números hiper-reais entre
$-1$ e $1$. O gráfico hiper-real de $\hyper{f}$, ilustrado na
Figura~\ref{fig:exhyperunitcirc}, pode ser traçado no papel desenhando-se
o gráfico real de $f(x)$ e apontando-se um microscópio infinitesimal para
alguns pontos chave.
\end{example}

\includefig[Gráfico hiper-real de $f(x) = \sqrt{1-x^2}$]{exhyperunitcirc}

\begin{example}
Seja $f$ a função identidade para os números reais, $f(x) = x$. Pelo
Axioma de Transferência, a equação $f(x) = x$ é verdadeira para todo
número hiper-real $x$. Assim, a extensão natural $\hyper{f}$ de $f$
está bem definida, e $\hyper{f}(x) = x$ para todo hiper-real $x$. A
Figura~\ref{fig:exhyperidentity} ilustra o gráfico hiper-real de
$\hyper{f}$. Sob um microscópio, a sua inclinação é de $45\degree$.
\end{example}

\includefig[Gráfico hiper-real de $y = x$]{exhyperidentity}

Vejamos a seguir um exemplo de uma função hiper-real que não é uma
extensão natural de uma função real.

\begin{example}
Uma função hiper-real, a qual já estudamos em detalhe, é a \emph{função
parte padronizada} $\st{x}$. Para números reais, a função parte padronizada
assume os mesmos valores que a função identidade, ou seja,
$$
  \st{x} = x \; \text{ para todo real } x.
$$
Entretando, o gráfico hiper-real de $\st{x}$, mostrado na
Figura~\ref{fig:exstandardpart}, é muito diferente do gráfico hiper-real
da função identidade $\hyper{f}$. O domínio da função parte padronizada
é o conjunto de todos os números finitos, enquanto que $\hyper{f}$ tem
como domínio $\setHR$. Logo, para $x$ infinito temos que $\hyper{f}(x) = x$
mas $\st{x}$ é indefinido. Se $x$ é finito mas não real, $\hyper{f}(x) = x$
mas $\st{x} \ne x$. Sob o miscroscópio, uma porção infinitesimal do gráfico
da função parte padronizada é horizontal, enquanto que a função identidade
tem inclinação de $45\degree$.

A função parte padronizada não é a extensão natural da função identidade,
nem sequer a extensão natural de alguma função real.
\end{example}

\includefig[Gráfico hiper-real de $y = \st{x}$]{exstandardpart}

A função parte padronizada é a única função hiper-real estudada neste
livro que não é a extensão natural de uma função real.

Concluímos com algumas palavras sobre a construção dos números reais e dos
numeros hiper-reais. Antes de Weierstrass, os números racionais já se
assentavam sobre uma base bem sólida, mas os números reais eram algo novo.
Antes que se pudesse usar os axiomas de números reais com segurança, era
necessário mostrar que esses axiomas não levavam a alguma contradição.
Isto foi feito a partir dos números racionais, construindo-se uma estrutura
que satisfazia todos os axiomas para os números reais. Como qualquer coisa
demonstrada a partir de axiomas é verdadeira nessa estrutura, os axiomas
não podem levar a uma contradição.

A ideia é construir números \emph{reais} a partir de sequências de Cauchy
de números \emph{racionais}.

\begin{defin}
\index{sequência!de Cauchy}
Uma \newdef[sequência!de Cauchy]{Sequência de Cauchy} é uma sequência de
números $\langle a_1, a_2, \ldots \rangle$ tal que, para cada real
$\epsilon > 0$ existe um inteiro $n$ tal que os números
$$
  \langle a_{n_\epsilon}, a_{n_\epsilon + 1}, a_{n_\epsilon + 2}, \ldots \rangle
$$
estão todos a uma distância máxima de $\epsilon$ entre eles.
\end{defin}

Duas sequências de Cauchy
$$
  \langle a_1, a_2, \ldots \rangle, \;\;\; \langle b_1, b_2, \ldots \rangle
$$
de números racionais são chamadas
\newdef[Cauchy equivalentes]{Cauchy equivalentes},
\index{equivalência!de Cauchy}
ou simbolicamente $\langle a_1, a_2, \ldots \rangle \equiv \langle b_1, b_2, \ldots \rangle$,
se a sequência das diferenças
$$\langle a_1 - b_1, a_2 - b_2, \ldots \rangle$$
converge a zero. Intuitivamente, isto significa que as duas sequências
possuem o mesmo limite.

\subpart{Propriedades da Equivalência de Cauchy}

\begin{enumerate}[(1)]
\item Se $\langle a_1 , a_2, \ldots \rangle \equiv \langle a'_1 , a'_2, \ldots \rangle$ e $\langle b_1 , b_2, \ldots \rangle \equiv \langle b'_1 , b'_2, \ldots \rangle$ então as sequências de somas são equivalentes,
$$
  \langle a_1+b_1 , a_2+b_2, \ldots \rangle \equiv \langle a'_1+b'_1 , a'_2+b'_2, \ldots \rangle.
$$
\item Sob as mesmas hipóteses, as sequências de produtos são equivalentes,
$$
  \langle a_1 \cdot b_1 , a_2 \cdot b_2, \ldots \rangle \equiv \langle a'_1 \cdot b'_1 , a'_2 \cdot b'_2, \ldots \rangle.
$$
\item Se $a_n = b_n$ para todo $n$, exceto, possivelmente, para uma quantidade
      finita de valores, então
      $$
        \langle a_1, a_2, \ldots \rangle \equiv
        \langle b_1, b_2, \ldots \rangle
      $$
\end{enumerate}

O conjunto de números reais é definido, portanto, como o conjunto de todas
as classes de equivalência de sequências de Cauchy de números racionais. Um
número racional $r$ corresponde à classe de equivalência da sequência
constante $\langle r, r, r, \ldots \rangle$. A soma entre a classe de
equivalência de $\langle a_1, a_2, \ldots \rangle$ e a classe de
equivalência de $\langle b_1, b_2, \ldots \rangle$ é definida como a classe
de equivalência da sequência de somas
$$
  \langle a_1+b_1 , a_2+b_2, \ldots \rangle.
$$
O produto é definido de forma similar. Pode-se demonstrar que todos os
axiomas para números reais são válidos para esta estrutura.

Hoje em dia, os números reais estão estabelecidos em bases bem sólidas,
enquanto que os números hiper-reais podem ser considerados uma nova
ideia. Robinson usou a construção de ultraproduto\index{ultraproduto}
de Skolem para demonstrar
que os axiomas para os números hiper-reais (por exemplo, como usados neste
livro) não levam a uma contradição. Os método é similar à construção dos
números reais a partir dos racionais. Contudo, desta vez, o sistema de
números reais é o ponto de partida. Construímos os números \emph{hiper-reais}
a partir de sequências \emph{arbitrárias} (não apenas de Cauchy) de números
\emph{reais}.

Uma \newdef[equivalência!de ultraproduto]{equivalência de ultraproduto}
é uma relação de equivalência $\equiv$ no conjunto de todas as sequências
de números reais que possuem as propriedades de Cauchy (1)--(3) e, além disso,
\begin{enumerate}[(1)]
\setcounter{enumi}{4}
\item Se cada $a_n$ pertence ao conjunto $\{0,1\}$, então
      $\langle a_1, a_2, \ldots \rangle$ é equivalente a exatamente uma das
      sequências constantes $\langle 0,0,0, \ldots \rangle$ ou
      $\langle 1,1,1,\ldots \rangle$.
\end{enumerate}

Dado uma relação de equivalência de ultraproduto, o conjunto dos números
hiper-reais é definido como o conjunto de todas as classes de equivalências
de sequências de números reais. Um número real $r$ corresponde à classe de
equivalência da sequência constante $\langle r,r,r,\ldots \rangle$. Somas
e produtos são definidos da mesma forma que para sequências de Cauchy.
A extensão natural $\hyper{f}$ de uma função real $f$ é definida de tal forma
que a imagem da classe de equivalência de $\langle a_1,a_2,\ldots \rangle$ é
a classe de equivalência de $\langle f(a_1),f(a_2),\ldots \rangle$. Pode-se
demonstrar que existem relações de equivalência de ultraproduto, e que todos
os axiomas para números reais e hiper-reais são válidos para a estrutura
definida desta forma.

Quando números hiper-reais são construidos como classes de equivalência de
de sequências de números reais, podemos dar exemplos específicos de números
hiper-reais infinitos. A classe de equivalência de
$$
  \langle 1,2,3,\ldots,n,\ldots \rangle
$$
é um número hiper-real infinito positivo. A classe de equivalência de
$$
  \langle 1,4,9,\ldots,n^9,\ldots \rangle
$$
é maior do que a anterior, e a classe de equivalencia de
$$
  \langle 1,2,4,\ldots,2^n,\ldots \rangle
$$
é um número hiper-real infinito ainda maior.

Podemos ainda dar exemplos de infinitésimos não nulos. As classes de
equivalência de
$$
  \begin{array}{rcl}
            & \langle 1,1/2,1/3,\ldots,1/n,\ldots \rangle & , \\
            & \langle 1,1/4,1/9,\ldots,1/n^2,\ldots \rangle & , \\
  \text{e } & \langle 1,1/2,1/4,\ldots,2^{-n},\ldots \rangle &
  \end{array}
$$
são infinitésimos positivos progressivamente menores.

O equívoco de Leibniz e seus contemporâneos foi identificar todos os
infinitésimos com o zero. Isto leva a uma contradição imediata, pois
$\dif{x}/\dif{y}$ torna-se $0/0$. De acordo com a presente abordagem,
as classes de equivalência de
$$
  \begin{array}{rcl}
            & \langle 1,1/2,1/3,\ldots,1/n,\ldots \rangle & , \\
  \text{e } & \langle 0, 0,  0, \ldots, 0, \ldots \rangle &
  \end{array}
$$
são números hiper-reais distintos. Eles não são iguais, apenas têm a
mesma parte padronizada, que é zero. Isto evita a contradição e, uma
vez mais, torna a abordagem por infinitésimos um método consistente
do ponto de vista matemático.

Para maiores informações sobre as ideias aborddas neste epílogo, veja
o suplemento do instrutor, \emph{Foundations of Infinitesimal Calculus}%
\tnote{ainda não traduzido para o português.}, que traz uma
abordagem autocontida de ultraprodutos e números hiper-reais.

\subpart{Para ler mais sobre a história do cálculo, veja:}

\begin{enumerate}[]
\item \emph{Tópicos de História da Matemática para Uso em Sala de Aula, vol. 6 -- Cálculo}; Carl B. Boyer, Atual Editora, São Paulo, 1993.
\item \emph{The History of the Calculus and its Conceptual Development}; Carl B. Boyer, Dover, Nova Iorque, 1949.
\item \emph{Mathematical Thought from Ancient to Modern Times}; Morris Kline, Oxford Univ. Press, Nova Iorque, 1972.
\item \emph{Non-standard Analysis}; Abraham Robinson, North-Holland, Amsterdã e Londres, 1966.
\end{enumerate}

\subpart{Para leituras mais avançadas sobre análise infinitesimal,
veja \emph{Non-standard Analysis} de Abraham Robinson e:}

\begin{enumerate}[]
\item \emph{Lectures on Non-standard Analysis}; M. Machover e J. Hirschfeld, Springer-Verlag, Berlim, Heidelberg, Nova Iorque, 1969.
\item \emph{Victoria Symposium on Nonstandard Analysis}; A. Hurd e P. Loeb, Springer-Verlag, Berlim, Heidelberg, Nova Iorque, 1973.
\item \emph{Studies in Model Theory}; M. Morley (editor), Mathematical Association of America, Providence, 1973.
\item \emph{Applied Nonstandard Analysis}; M. Davis, Wiley, Nova Iorque, 1977.
\item \emph{Introduction to the Theory of Infinitesimals}; K. D. Stroyan e W. A. Luxemburg, Academic Press, Nova Iorque e Londres, 1976.
\item \emph{Foundations of Infinitesimal Stochastic Analysis}; K. D. Stroyan e J. M. Bayod (editores). In: Studies in Logic and the Foundations of Mathematics, vol. 119, North-Holland, Amsterdã e Londres, 1986.
\end{enumerate}

