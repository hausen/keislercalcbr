% Front matter
\frontmatter

% r.1 blank page
\newpage

\setcounter{page}{3}

Dedicado a meus filhos, Randall, Jeffrey, e Thomas.

\hfill H. Jerome Keisler

\vfill

\emph{Haters gonna hate.}

\hfill Autor desconhecido

\vfill

Copyright © 2000 by H. Jerome Keisler.

Versão traduzida a partir da segunda edição americana, revista
em fevereiro de 2012. Traduzida em 2014 por Rodrigo Hausen.

Este trabalho está registrado sob a licença Creative Commons
por Atribuição-NãoComercial-CompartilhaIgual 3.0, de acordo
com o desejo original do autor. Para ver uma cópia desta licença,
visite \url{https://creativecommons.org/licenses/by-nc-sa/3.0/br/}
ou envie uma carta a Creative Commons, 559 Nathan
Abbott Way, Stanford, California, 93405, Estados Unidos.

\newpage

\chapter*{Prefácio à versão brasileira}

Esta versão foi preparada pois atualmente, na segunda década do século XXI,
nos deparamos com uma quase total ausência de materiais em língua portuguesa
sobre cálculo pela abordagem de infinitésimos e análise hiper-real, por
vezes também chamada de ``não padronizada.''

Tentamos nos manter fiéis ao texto original, mas como este
possui mais de três décadas e foi idealizado para um público ligeiramente
diverso do nosso, algumas modificações se fizeram necessárias.

Para alguns termos usados à época, expressões aproximadas em português
tiveram de ser escolhidas. Um exemplo disso é
o termo ``\emph{range},'' antigamente usado para, dependendo
do contexto, denominar o contradomínio ou a imagem de uma função.
Optamos por escrevê-lo como ``imagem'' nesta versão em português.

Além disso, escolhemos uma notação ligeiramente diversa da usada no
original para aproximá-la das notações usadas atualmente no Brasil,
e também para evitar confusão com notações cujo uso mais comum
nos dias de hoje indica outro significado.
Por exemplo, enquanto que no original o conjunto dos reais é
denotado simplesmente $R$, e o conjunto dos hiper-reais é
denotado $R^*$, optamos por usar respectivamente os símbolos
$\setR$ e $\setHR$.

Foi também escolha do tradutor substituir o uso do asterisco ($*$)
pela cerquilha ($\#$) para denotar as extensões de reais aos hiper-reais.
Com esta substituição, procuramos evitar a confusão com a notação de
exclusão do zero de um conjunto numérico. A escolha pelo símbolo
$\#$ deu-se por este se assemelhar ao símbolo de sustenido em
música, que eleva uma nota musical em um semitom; em um piano,
os sustenidos (teclas pretas) encontram-se entre as notas musicais
regulares (teclas brancas). Desta forma, sugerimos que os
hiper-reais são um conjunto distinto dos reais, tendo uma infinidade
de ``sustenidos,'' números hiper-reais que não são reais, entre as
``notas regulares,'' os números reais.

\hfill Rodrigo Hausen

\newpage

\chapter*{Prefácio à segunda edição americana}

Nesta segunda edição, várias mudanças foram feitas baseadas em nove
anos de experiência em sala de aula. Há revisões substanciais aos seis
primeiros capítulos e ao Epílogo, e há um capítulo completamente novo,
Capítulo 14, sobre equações diferenciais. Além disso, os Capítulos 11
e 12 originais foram reorganizados como três capítulos: Capítulo 11
sobre diferenciação parcial, Capítulo 12 sobre integração múltipla, e
Capítulo 13 sobre cálculo vetorial.

O Capítulo 1 foi encurtado, e a maior parte do material teórico da
primeira edição foi movida para o Epílogo. O cálculo de funções
transcendentais foi totalmente integrado ao curso, começando pelo
Capítulo 2 sobre derivadas. O Capítulo 3 foca em aplicações de
derivadas. O material sobre problemas com palavras e sobre taxas
relacionadas foi movido dos dois primeiros capítulos para o começo
do Capítulo 3. Os resultados teóricos sobre funções contínuas,
incluindo os Teoremas do Valor Intermediário, Extremo e Médio,
foram compilados em uma única seção ao final do Capítulo 2. O
desenvolvimenot da integral no Capítulo 4 foi simplificado. A 
%% TODO: "regra do trapézio"?
Regra do Trapézio foi trazida do Capítulo 5 para o Capítulo 4, e
uma discussão da Regra de Simpson foi adicionada. A seção sobre a
área entre duas curvas foi movida do Capítulo 4 para o Capítulo 4.
O Capítulo 5 lida com limites, aproximações e geometria analítica.
Um tratamento amplo às seções cônicas e uma seção sobre o método de
Newton foram adicionados. O Capítulo 6 começa com um novo material
sobre determinação de volume por integração de áreas de seções
transversais.

Apenas pequenas mudanças e correções foram feitas nos Capítulos 7
a 13. O novo Capítulo 14 fornece uma primeira introdução a equações
diferenciais, com ênfase na solução de equações diferenciais lineares
de primeira e segunda ordens. Na Seção 14.2, infinitésimos são usados
para dar uma prova simples à afirmação de que toda equação diferencial
$y' = f(t,y)$, onde $f$ é contínua, possui uma solução. A prova deste
fato está além do escopo de um curso de cálculo elementar, mas é alcançável
quando se usam infinitésimos.

Desejo agradecer todos os meus amigos e colegas que sugeriram correções
e aperfeiçoamentos à primeira edição deste livro. 

\hfill H. Jerome Keisler

\chapter*{Prefácio à primeira edição americana}

O cálculo foi originalmente desenvolvido usando-se o conceito intuitivo
de infinitésimo, ou um número infinitamente pequeno. Porém, nos últimos
cem anos, infinitésimos foram banidos do curso de cálculo por razões de
rigor matemático. Estudantes tiveram que aprender o assunto sem a
intuição original. Este livro de cálculo é baseado no trabalho de
Abraham Robinson, que em 1960 encontrou um modo de tratar com rigor os
infinitésimos. Enquanto que o curso tradicional começa com o conceito
difícil de limite, este curso inicia-se com infinitésimos, que são mais
fáceis de se compreender. Destina-se ao estudante médio que inicia em
cálculo e cobre a sequência usual de três ou quatro semestres.

A abordagem por infinitésimos tem três vantagens importantes para o
estudante. Primeiramente, é mais próxima da intuição que levou originalmente
ao cálculo. Em segundo lugar, os conceitos centrais de derivada e integral
tornam-se mais acessíveis a compreensão e uso pelo estudante. Por último,
ela ensina ambas as abordagens por infinitésimos e tradicional, dando ao
estudante uma ferramenta extra, a qual pode se tornar cada vez mais
importante no futuro.

Antes de descrever este livro, eu gostaria de inserir o trabalho de A.
Robinson no contexto histórico. Na década de 1670, Leibinitz e Newton
desenvolveram o cálculo com base na noção intuitiva de infinitésimos.
Os infinitésimos foram usados por mais duzentos anos, até que o
primeiro tratamento rigorozo do cálculo foi aperfeiçoado por
Weierstrass na década de 1870. O curso padrão de cálculo dos dias
de hoje ainda é baseado nas definições dadas por Weierstrass para o limite
``usando epsilons e deltas.'' Em 1960, Robinson resolveu um
problema de trezentos anos ao dar um tratamento preciso ao cálculo
usando infinitésimos. A conquista de Robinson provavelmente será
considerada um dos maiores avanços matemáticos do século XX.

Recentemente, infinitésimos tiveram empolgantes aplicações fora
da matemática, notadamente nos campos de economia e física. Como
é natural o uso de infinitésimos na modelagem de processos físicos
e sociais, tais aplicações certamente vão crescer em variedade
e importância. Esta é uma oportunidade única para encontrarmos
novos usos para a matemática, mas poucas pessoas foram atualmente
preparadas por meio de um treinamento que tire vantagem desta
oportunidade.

Sendo nova esta abordagem ao cálculo, alguns instrutores podem
precisar de materiais suplementares de apoio. Um volume do instrutor,
``\emph{Foundations of Infinitesimal Calculus}%
\tnote{Ainda não traduzido para o português.}%
,'' provê o suporte necessário e desenvolve a teoria em detalhe. O
volume do instrutor é atrelado a este livro, mas é autocontido
e destina-se ao público matemático geral.

Este livro contém todos os tópicos comuns de cálculo, incluindo a
definição tradicional de limite, além de uma ferramenta adicional
-- os infinitésimos. Assim, o estudante estará preparado para
cursos mais avançados da maneira que eles são ensinados atualmente.
Nos Capítulos de 1 até 4 os conceitos básicos de derivada, continuidade
e integral são desenvolvidos rapidamente pelo uso de infinitésimos. O
conceito tradicional de limite é adiado até o Capítulo 5, onde ele
é motivado por problemas de aproximação. Os últimos capítulos desenvolvem
funções transcendentais, séries, vetores, derivadas parciais e integrais
múltiplas. A teoria difere de um curso tradicional, mas a notação e os
métodos de solução de problemas são os mesmos na prática. Há uma variedade
de aplicações tanto a ciências naturais quanto sociais.

Eu incluí a seguinte inovação para instrutores que desejam introduzir
funções transcendentais cedo em seus cursos. No fim do Capítulo 2 sobre
derivadas, há um seção que começa um caminho alternativo sobre funções
transcendentais, e cada um dos Capítulos entre o 3 e o 6 possuem
caminhos alternativos com conjuntos de problemas sobre funções
transcendentais. Esta rota alternativa pode ser usada para fornecer
uma maior variedade nos problemas iniciais, ou pode ser omitida de
forma a se chegar às integrais o mais rápido possível. Nos Capítulos
7 e 8, as funções transcendentais são novamente desenvolvidas em um
ritmo mais lento.

Este livro é escrito para o estudante médio. Os problemas precedidos pelo
símbolo de um quadrado vão além dos exemplos trabalhados no texto e
destinam-se aos mais aventurosos.

Eu fui originalmente guiado a escrever este livro quando se tornou claro
para mim que o cálculo infinitesimal de Robinson poderia se tornar
disponível aos alunos de primeiro ano de um curso superior. A teoria é
apresentada de maneira simples; por exemplo, o trabalho de Robinson
usava lógica matemática, mas não este livro. Eu usei uma versão preliminar
deste livro em um curso de um semestre na Universidade de Wisconsin em
1969. Em 1971, foi publicada uma versão experimental para um curso em
dois semestres. Ela foi usada em várias faculdades e no colégio Nicolet,
próximo de Milwaukee, e foi testada em cinco escolas em um experimento
controlado, liderado pela Irmã Kathleen Sullivan entre 1972 e 1974. Os
resultados (publicados em 1974 na sua tese de doutoramento pela Universidade
do Wisconsin) mostram a viabilidade da abordagem por infinitésimos e serão
sumariados em um artigo no periódico American Mathematical Monthly\tnote{Sullivan, K. ``The Teaching of Elementary Calculus Using the Nonstandard Analysis Approach'', \emph{The American Mathematical Monthly}, v. 83, n. 5, 370--375, 1976.}.

Encontro-me em dívida com muitos colegas e estudantes que me encorajaram,
me deram conselhos e que cuidadosamente leram e usaram este manuscrito
em seus vários estágios de preparação. Devo agradecimentos especiais a
Jon Barwise, Universidade de Wisconsin; G. R. Blakley, Universidade de
Texas A \& M; Kenneth A. Bowen, Universidade de Syracuse; William P.
Francis, Universidade Tecnológica de Michigan; A. W. M. Glass,
Universidade de Bowling Green; Peter Loeb, Universidade de Illinois em
Urbana; Eugene Madison e Keith Stroyan, Faculdade Barat; e Frank Wattenberg,
Universidade de Massachusetts.

\hfill H. Jerome Keisler


