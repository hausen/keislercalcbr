\newcommand{\tnote}[1]{%
\footnote{N. do T.: #1}%
}

\newcounter{exerciseNo}

\newcommand{\exer}[1]{%

\noindent%
\begin{minipage}[t]{0.1\textwidth}
\hspace{3ex}\addtocounter{exerciseNo}{1}\textbf{\arabic{exerciseNo}}
\end{minipage}%
\begin{minipage}[t]{0.9\textwidth}
#1
\end{minipage}\\[-10pt]

}

\newcommand{\hardex}[1]{%

\noindent%
\begin{minipage}[t]{0.1\textwidth}
\begin{picture}(0,0)$\square$\end{picture}\hspace{3ex}\addtocounter{exerciseNo}{1}\textbf{\arabic{exerciseNo}}
\end{minipage}%
\begin{minipage}[t]{0.9\textwidth}
#1
\end{minipage}

}

\newcommand{\sectionproblems}{%
\section*{Problemas Para a Seção \the\value{chapter}.\the\value{section}}
\setcounter{exerciseNo}{0}
}

\newcommand{\chapterproblems}{%
\section*{Problemas Adicionais para o Capítulo~\the\value{chapter}}
\addcontentsline{toc}{section}{Problemas Adicionais para o Capítulo~\the\value{chapter}}
\setcounter{exerciseNo}{0}
}


\usepackage{ifthen}

\newcommand{\newdef}[2][]{%
\ifthenelse{\equal{#1}{}}%
{\emph{#2}\index{#2}}%
{\emph{#2}\index{#1}}%
}
\newcommand{\hyper}[1]{{#1}^\#}

\def\setR{\mathbb{R}}
\def\setHR{\hyper{\setR}}
\def\setN{\mathbb{N}}
\def\setZ{\mathbb{Z}}
\def\setQ{\mathbb{Q}}

\def\lth{\hyper{<}}
\def\definitionname{Definição}
\def\theoremname{Teorema}
\def\examplename{Exemplo}
\spnewtheorem*{defin}{Definição}{\bfseries}{\rmfamily}
\spnewtheorem*{theor}{Teorema}{\bfseries}{\rmfamily}

\newcommand\subpart[1]{

\vspace{\baselineskip}
\noindent\textbf{#1}
\vspace{\baselineskip}

}

% \narrowfigure[caption]{label}{commands}
\newcommand{\narrowfigure}[3][]{
\begin{figure}
#3
\caption{#1}
\label{#2}
\end{figure}
}

% \widefigure[caption]{label}{commands}
\newcommand{\widefigure}[3][]{
\narrowfigure[#1]{#2}{\begin{center}#3\end{center}}
}

% \narrowfigurefile[caption]{label}{path}
\newcommand{\narrowfigurefile}[3][]{
\narrowfigure[#1]{#2}{\includegraphics{#3}}
}

\newcommand{\widefigurefile}[3][]{
\narrowfigure[#1]{#2}{\begin{center}\includegraphics{#3}\end{center}}
}

% \narrowtable[caption]{label}{coldef}{table}
\newcommand{\narrowtable}[4][]{
\begin{table}
\caption{#1}
\label{#2}
\begin{tabular}{#3}
#4
\end{tabular}
\end{table}
}

% \widetable[caption]{label}{coldef}{table}
\newcommand{\widetable}[4][]{
\begin{table}
\caption{#1}
\label{#2}
\begin{center}
\begin{tabular}{#3}
#4
\end{tabular}
\end{center}
\end{table}
}


\DeclareMathOperator{\sen}{sen}
