\newcommand{\tnote}[1]{%
\footnote{N. do T.: #1}%
}

\newcommand{\problemsection}[1]{%
\section*{Problemas Adicionais para o Capítulo~#1}
\addcontentsline{toc}{section}{Problemas Adicionais para o Capítulo~#1}
}

\usepackage{ifthen}

\newcommand{\newdef}[2][]{%
\ifthenelse{\equal{#1}{}}%
{\emph{#2}\index{#2}}%
{\emph{#2}\index{#1}}%
}
\newcommand{\hyper}[1]{{#1}^\#}

\def\setR{\mathbb{R}}
\def\setHR{\hyper{\setR}}
\def\setN{\mathbb{N}}
\def\setZ{\mathbb{Z}}
\def\setQ{\mathbb{Q}}

\def\lth{\hyper{<}}
\def\definitionname{Definição}
\def\theoremname{Teorema}
\spnewtheorem*{defin}{Definição}{\bfseries}{\rmfamily}
\spnewtheorem*{theor}{Teorema}{\bfseries}{\rmfamily}

\newcommand\subpart[1]{

\vspace{\baselineskip}
\noindent\textbf{#1}
\vspace{\baselineskip}

}

\DeclareMathOperator{\sen}{sen}
