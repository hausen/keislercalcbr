\graphicspath{ {./figuras/02_diff/} }
\chapter{Diferenciação}
\label{chp:diff}

\section{Derivadas}
\label{sec:derivatives}

Estamos prontos para explicar o que significa a inclinação de uma curva
ou a velocidade de um ponto em movimento. Considere uma função real $f$
e um número real $a$ no domínio de $f$. Quando $x$ assume o valor $a$,
a imagem $f(x)$ assume o valor $f(a)$. Agora, suponha que o valor de $x$
é alterado de $a$ para um número hiper-real $a + \Delta{x}$, o qual está
infinitamente próximo mas não é igual a $a$. Então o novo valor de $f(x)$
será $f(a+\Delta{x})$. Neste processo, o valor de $x$ será alterado por
uma quantidade infinitesimal não nula $\Delta{x}$, enquanto que a mudança
no valor de $f(x)$ será
$$
  f(a+\Delta{x})-f(a).
$$
A razão entre a mudança no valor de $f(x)$ e a mudança no valor de $x$ é
$$
  \frac{f(a+\Delta{x}-f(a)}{\Delta{x}}.
$$
Esta razão é usada na definição da inclinação de $f$, a qual daremos agora.

\begin{defin}
Dizemos que $S$ é a \newdef{inclinação} de $f$ em $a$ se
$$
  S = \st{\frac{f(a+\Delta{x})-f(a)}{\Delta{x}}}.
$$
para todo infinitésimo não nulo $\Delta{x}$.
\end{defin}

A inclinação, quando existe, está infinitamente próxima para a razão entre a
mudança em $f(x)$ e uma mudança infinitamente pequena em $x$. Dada uma curva
$y = f(x)$, a inclinação de $f$ em $a$ é também chamada de inclinação da
curva $y=f(x)$ em $x=a$. A Figura~\ref{fig:slopehyperrealline} mostra um
infinitésimo não nulo $\Delta{x}$ e uma reta hiper-real por dois pontos na
curva, em $a$ e em $a + \Delta{x}$. A quantidade
$$
  \frac{f(a+\Delta{x})-f(a)}{\Delta{x}}
$$
é a inclinação desta reta, e sua parte padronizada é a inclinação da curva.

\includefig{slopehyperrealline}

A inclinação de $f$ em $a$ nem sempre existe. Temos a seguir uma lista de
todas as possibilidades.

\begin{enumerate}[(1)]
\item A inclinação de $f$ em $a$ existe se a razão
$$
  \frac{f(a+\Delta{x})-f(a)}{\Delta{x}}
$$
é finita e possui a mesma parte padronizada para todo infinitésimo
$\Delta{x} \ne 0$. Ela tem valor
$$
  S = \st{\frac{f(a+\Delta{x})-f(a)}{\Delta{x}}}.
$$

\item A inclinação de $f$ em $a$ não existe em quaisquer um dos
      quatro casos a seguir:
      \begin{enumerate}[(a)]
        \item $f(a)$ não está definido.
        \item $f(a+\Delta{x})$ não está definido para algum infinitésimo
              $\Delta{x} \ne 0$.
        \item O termo $\displaystyle \frac{f(a+\Delta{x})-f(a)}{\Delta{x}}$
              é infinito para algum infinitésimo $\Delta{x} \ne 0$.
        \item A parte padronizada de $\displaystyle \frac{f(a+\Delta{x})-f(a)}{\Delta{x}}$
              assume valores diferentes para infinitésimos distintos
              $\Delta{x} \ne 0$.
      \end{enumerate}
\end{enumerate}

Podemos considerar a inclinação de $f$ em qualquer ponto $x$, o que nos
dá uma nova função de $x$.

\begin{defin}
Sja $f$ uma função real de uma variável. A \newdef{derivada} de $f$ é a
nova função $f'$ cujo valor em $x$ é a inclinação de $f$ em $x$.
Simbolicamente,
$$
  f'(x) = \st{\frac{f(x+\Delta{x})-f(x)}{\Delta{x}}},
$$
sempre que a inclinação existe.
\end{defin}

O valor da derivada $f'(x)$ é indefinido se a inclinação de $f$ não
existe em $x$.

Para um dado ponto $a$, a inclinação de $f$ em $a$ e a derivada de $f$ em
$a$ são a mesma coisa. Geralmente usamos a palavra ``inclinação'' para
destacar a intuição geométrica, e usamos ``derivada'' para ressaltar o
fato que $f'$ é uma função.

O processo de se encontrar a derivada de $f$ é chamado \newdef{differenciação}
ou \newdef{derivação}. Dizemos que $f$ é \newdef{diferenciável}, ou
\newdef{derivável}, em $a$ se o valor $f'(a)$ está definido; ou seja,
se existe inclinação de $f$ em $a$.

Variáveis independentes e dependentes são úteis no estudo de derivadas.
Vamos recapitular brevemente o que são. Um \newdef{sistema de fórmulas}
é um conjunto finito de equações e inequações. Dado um sistema de fórmulas,
cujo gráfico é o mesmo da equação simples $y = f(x)$, dizemos que
\emph{$y$ é uma função de $x$}, ou que \emph{$y$ depende de $x$}, e chamamos
$x$ a \newdef[variável!independente]{variável independente} e $y$ a
\newdef[variável!dependente]{variável dependente}.

Quando $y = f(x)$, podemos introduzir uma nova variável independente
$\Delta{x}$ e uma nova variável dependente $\Delta{y}$, com a equação
\begin{eqnarray}
  \Delta{y} & = & f(x+\Delta{x}) - f(x). \label{eq:chpdiff:deltay}
\end{eqnarray}

Esta equação determina $\Delta{y}$ como uma função real de duas variáveis
$x$ e $\Delta{x}$, quando $x$ e $\Delta{x}$ variam entre os números reais.
Geralmente, desejaremos usar a Equação~\ref{eq:chpdiff:deltay} para
$\Delta{y}$ quando $x$ é um numero real e $\Delta{x}$ é um infinitésimo
não nulo. O Princípio da Transferência implica que a
Equação~\ref{eq:chpdiff:deltay} também determina $\Delta{y}$ como uma
função hiper-real de duas variáveis quando permitimos que $x$ e $\Delta{x}$
variem entre os números hiper-reais.

O valor $\Delta{y}$ é chamado de \newdef{incremento} de $y$. Geometricamente,
o incremento $\Delta{y}$ é a variação em $y$ ao longo da curva, correspondendo
a uma variação de $\Delta{x}$ em $x$. O símbolo $y'$ é por vezes usado para
a derivada $y' = f'(x)$. Assim, a equação hiper-real
$$
  f'(x) = \st{\frac{f(x+\Delta{x})-f(x)}{\Delta{x}}}
$$
pode assumir a forma abreviada
$$
  y' = \st{\frac{\Delta{y}}{\Delta{x}}}.
$$

O infinitésimo $\Delta{x}$ pode ser tanto positivo quanto negativo, mas
nunca pode ser zero. As várias possibilidades são ilustradas na
Figura~\ref{fig:slopecases} com o uso de um microscópio infinitesimal.
Os sinais de $\Delta{x}$ e $\Delta{y}$ são indicados nas legendas.

\includefig{slopecases}

Nossas regras para partes padronizadas podem ser usadas em muitos casos
para encontrarmos a derivada de uma função. Há duas partes no problema de
se encontrar a derivada $f'$ de uma função $f$:
\begin{enumerate}[(1)]
\item Encontre o domínio de $f'$.
\item Encontre o valor de $f'(x)$ quando este for definido.
\end{enumerate}

\begin{example}
Encontre a derivada da função
$$
  f(x) = x^3.
$$
Neste exemplo, assim como nos próximos, deixaremos $x$ variar entre os números
reais e $\Delta{x}$ variar entre os infinitésimos não nulos. Vamos introduzir
a nova variável $y$ com a equação $y=x^3$. Primeiramente, determinamos
$\Delta{y}/\Delta{x}$.
\begin{eqnarray*}
  y           & = & x^3, \\
  y+\Delta{y} & = & (x+\Delta{x})^3, \\
  \Delta{y}   & = & (x+\Delta{x})^3 - x^3, \\
  \frac{\Delta{y}}{\Delta{x}} & = &
    \frac{(x+\Delta{x})^3 - x^3}{\Delta{x}}.
\end{eqnarray*}
A seguir, simplificamos a expressão para $\Delta{y}/\Delta{x}$.
\begin{eqnarray*}
  \frac{\Delta{y}}{\Delta{x}} & = &
    \frac{(x^3+3x^2\Delta{x}+3x(\Delta{x})^2+(\Delta{x})^3)-x^3}
         {\Delta{x}} \\
  & = &
    \frac{3x^2\Delta{x}+3x(\Delta{x})^2+(\Delta{x})^3}
         {\Delta{x}} \\
  & = &
    3x^2+3x\Delta{x}+(\Delta{x})^2
\end{eqnarray*}
Então, tomamos a parte padronizada,
\begin{eqnarray*}
  \st{\frac{\Delta{y}}{\Delta{x}}} & = &
    \st{3x^2+3x\Delta{x}+(\Delta{x})^2} \\
  & = &
    \std(3x^2)+\std(3x\Delta{x})+\std((\Delta{x})^2) \\
  & = & 3x^2 + 0 + 0 = 3x^2.
\end{eqnarray*}
Portanto, $f'(x) = \st{\frac{\Delta{y}}{\Delta{x}}} = 3x^2.$
\end{example}

Demonstramos que a derivada da função $f(x) = x^3$ é a função $f'(x) = 3x^2$,
cujo domínio é toda a reta real. Os gráficos de $f$ e $f'$ são mostrados na
Figura~\ref{fig:cap2ex1}.

\includefig{cap2ex1}

\begin{example}
Determine $f'(x)$, onde $f(x) = \sqrt{x}$.
\begin{caseanalysis}
\item $x < 0$. Como $\sqrt{x}$ não está definido, $f'(x)$ não existe.
\item $x = 0$. Quando $\Delta{x}$ é um infinitésimo negativo, o termo
      $$
        \frac{\sqrt{x+\Delta{x}}-\sqrt{x}}{\Delta{x}} =
          \frac{\sqrt{0+\Delta{x}}-\sqrt{0}}{\Delta{x}}
      $$
      não está definido pois $\sqrt{\Delta{x}}$ é indefinido. Quando
      $\Delta{x}$ é um infinitésimo positivo, o termo
      $$
        \frac{\sqrt{x+\Delta{x}}-\sqrt{x}}{\Delta{x}} =
          \frac{\sqrt{\Delta{x}}}{\Delta{x}} =
            \frac{1}{\sqrt{\Delta{x}}}
      $$
      está definido, mas é um hiper-real infinito. Por duas razões
      (apenas uma delas bastaria), $f'(x)$ não existe.
\item $x>0$. Seja $y=\sqrt{x}$. Então
      \begin{eqnarray*}
        y+\Delta{y}                 & = & \sqrt{x + \Delta{x}}, \\
        \Delta{y}                   & = & \sqrt{x + \Delta{x}} - \sqrt{x}, \\
        \frac{\Delta{y}}{\Delta{x}} & = &
          \frac{\sqrt{x + \Delta{x}} - \sqrt{x}}{\Delta{x}}.
      \end{eqnarray*}
      Calculemos agora
      \begin{eqnarray*}
        \frac{\Delta{y}}{\Delta{x}} & = &
          \frac{\sqrt{x + \Delta{x}} - \sqrt{x}}{\Delta{x}} \cdot
          \frac{(\sqrt{x + \Delta{x}} + \sqrt{x})}
               {(\sqrt{x + \Delta{x}} + \sqrt{x})} \\
        & = &
          \frac{(x+\Delta{x})-x}{\Delta{x}(\sqrt{x + \Delta{x}} + \sqrt{x})} \\
        & = &
          \frac{\Delta{x}}{\Delta{x}(\sqrt{x + \Delta{x}} + \sqrt{x})} =
          \frac{1}{\sqrt{x + \Delta{x}} + \sqrt{x}}.
      \end{eqnarray*}
      Tomando partes padronizadas,
      \begin{eqnarray*}
        \st{\frac{\Delta{y}}{\Delta{x}}} & = &
          \st{\frac{1}{\sqrt{x + \Delta{x}} + \sqrt{x}}} \\
        & = &
          \frac{1}{\std(\sqrt{x + \Delta{x}} + \sqrt{x})} \\
        & = &
          \frac{1}{\std(\sqrt{x + \Delta{x}}) + \std(\sqrt{x})} \\
        & = &
          \frac{1}{\sqrt{x} + \sqrt{x}} = \frac{1}{2\sqrt{x}}.
      \end{eqnarray*}
      Portanto, quando $x > 0$,
      $$
        f'(x) = \frac{1}{2\sqrt{x}}.
      $$
\end{caseanalysis}
Assim, a derivada de $f(x) = \sqrt{x}$ é a função
$f'(x) = \frac{1}{2\sqrt{x}}$, e seu domínio é o conjunto de todos os
$x$ tais que $x > 0$ (veja Figura~\ref{fig:cap2ex2}).
\end{example}

\includefig{cap2ex2}

\section{Diferenciais e Retas Tangentes}
\label{sec:tglines}

\section{Derivadas de Funções Racionais}
\label{sec:derivratfunc}

\section{Funções Transcendentais}
\label{sec:transcfunc}

\section{Regra da Cadeia}
\label{sec:chainrule}

\section{Derivadas de Ordem Superior}
\label{sec:higherderivs}

\section{Funções Implícitas}
\label{sec:implicitfunc}

\begin{chapterproblems}
\end{chapterproblems}


