\pagestyle{empty}

\setlength{\parindent}{0pt}
\setlength{\parskip}{2pt}

\textsc{Regras de Integração}

{\small
\begin{multicols}{2}
$\displaystyle \int du = u + C$

$\displaystyle \int du + dv = \int u +\int dv$ (Regra da Soma)

$\displaystyle \int k \; du = k \int du$ (Regra da Constante)

$\displaystyle \int u \; dv = uv - \int v \; du$ (Integração por Partes)
\end{multicols}
}

\textsc{Tabela de Integrais}

\def\dx{\;dx}

{\small
(A constante de integração foi omitida.)

\begin{multicols}{2}
$\displaystyle \int x^r \dx = \frac{x^{r+1}}{r+1}, r \ne -1$

$\displaystyle \int e^x \dx = e^x$

$\displaystyle \int \sen x \dx = - \cos x$

$\displaystyle \int \tan x \dx = \ln|\sec x|$
\end{multicols}
}

\newpage

\textsc{Álgebra Elementar dos Números Reais}

\begin{multicols}{2}
$a + b = b + a$

$a + (b + c) = (a + b) + c$

$a(b+c) = ab+ac$

$a + 0 = a \cdot 1 = a$

$a - a = 0$

$-(-a) = a$

$-(a-b) = b - a$

$\displaystyle \frac{a}{b} + \frac{c}{d} = \frac{ad + bc}{bd}$ $(b,d \ne 0)$

$a^1 = a$

$1^a = a$

$a^{m+n} = a^m a^n$

$\displaystyle a^{m-n} = \frac{a^m}{a^n}$ $(a \ne 0)$

$a^m b^m = (ab)^m$

$\displaystyle \sqrt[n]{-a} = -\sqrt[n]{a}, (-a)^{m/n} = \sqrt[n]{(-a)^m},$ ($a > 0$,\\
\hspace*{\fill} $n$ ímpar) \hspace*{2ex}

Se $a < b$, então $a + c < b + c$.

Se $a < b$, então $-b < -a$.

$ab = ba$

$a(bc) = (ab)c$

$a(-b) = (-a)b = -ab$

$a \cdot 0 = 0$

$a/a = 1$ ($a \ne 0$)

$\displaystyle \frac{1}{1/a} = a$ ($a \ne 0$)

$\displaystyle \frac{1}{a/b} = \frac{b}{a}$ ($a,b \ne 0$)

$\displaystyle \frac{a}{b} \cdot \frac{c}{d} = \frac{ac}{bd}$ ($b,d \ne 0$)

$\displaystyle a^0 = 1$ ($a \ne 0$)

$\displaystyle 0^n = 0$ ($n > 0$)

$\displaystyle a^{mn} = \left(a^m\right)^n$

$\displaystyle a^{m/n} = \sqrt[n]{a^m}$ ($a > 0$)

$\displaystyle \sqrt[m]{a} \sqrt[m]{b} = \sqrt[m]{ab}$ ($a,b > 0$)

Se $a < b$ e $0 < c$, então $ac < bc$.

Se $0 < a < b$, então $1/b < 1/a$.
\end{multicols}

$|-a| = |a|$ \hspace{4ex} $|a|\cdot|b| = |ab|$ \hspace{4ex}
$\displaystyle \frac{|a|}{|b|} = \left| \frac{a}{b} \right|$ ($b \ne 0$)

As seguintes formas são indefinidas:
$a/0$, $0^0$, $0^{-m}$, $\sqrt[n]{-a}$ ($a,m$ positivos, $n$ par)

\emph{Fórmula Quadrática}: $\displaystyle a x^2 + b x + c = 0$ se, e somente se,
$\displaystyle x = \frac{-b \pm \sqrt{b^2 - 4ac}}{2a}$


\textsc{Álgebra dos Números Hiper-reais}

Notação:
\begin{itemize}
\item $\epsilon, \delta$ são infinitesimais positivos
\item $b,c$ são positivos e finitos, mas não infinitesimais
\item $H,K$ são positivos infinitos
\end{itemize}

Os seguintes são infinitesimais:
\[
 -\epsilon, \SPC  1/H, \SPC  \epsilon/b, \SPC  \epsilon/H, \SPC  b/H, \SPC  \epsilon + \delta, \SPC  \epsilon - \delta, \SPC  \epsilon \cdot \delta, \SPC  b \cdot \epsilon, \SPC  \sqrt[n]{\epsilon}
\]

Os seguintes são finitos, mas não infinitesimais:
\[
 -b, \SPC 1/b, \SPC b/c, \SPC b + \epsilon, \SPC b \cdot c, \SPC \sqrt[n]{b},
 \SPC b+c
\]
$b-c$ é finito (possivelmente infinitesimal)

Os seguintes são infinitos:
\[
 -H, \SPC  1/\epsilon, \SPC  b/\epsilon, \SPC  H/\epsilon, \SPC  H/b, \SPC  H+\epsilon, \SPC  H+b, \SPC  H\cdot K, \SPC  \sqrt[n]{H}, \SPC  H+K
\]

Cada um destes pode ser infinitesimal, finito mas não infinitesimal,
ou infinito:
\[
 \epsilon/\delta, \SPC H/K, \SPC H\epsilon, \SPC H - K
\]

\textsc{Partes Padronizadas}

Considere $b$, $c$ finitos (possivelmente infinitesimais),
$\epsilon$ infinitesimal e $H$ infinito.

\begin{multicols}{2}
$\st{b + c} = \st{b} + \st{c}$

$\st{b c} = \st{b} \st{c}$

$\st{\sqrt[n]{b}} = \sqrt[n]{\st{b}}$ se $b > 0$ e $n > 0$

$b \approx \st{b}$

$b = \st{b}$ se, e somente se, $b$ é real

$\st{\epsilon} = 0$, \SPC $\st{H}$ é indefinido

$\st{b - c} = \st{b} - \st{c}$

$\st{b/c} = \st{b} / \st{c}$ se $\st{c} \ne 0$

$\st{b^c} = \st{b}^{\st{c}}$ se $\st{b} > 0$

$b \approx c$ se, e somente se, $\st{b} = \st{c}$

se $b \le c$, então $\st{b} \le \st{c}$\\
\end{multicols}

